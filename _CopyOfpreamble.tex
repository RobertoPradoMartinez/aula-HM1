%% MODELO DE MAQUETACION CON BOOKDOWN
%% ==================================
%% Preámbulo LaTeX para maquetación de traballos con RStudio
%% Personalizado a partires do modelo de apuntes de Historia da Música.
%% 
\usepackage{booktabs}
%%
%\usepackage[utf8]{inputenc} % acentos en ES (da erro ao compilar en R)
%\usepackage{cite}           % Citas Bibtex (da erro ao compilar en R)
%\usepackage[natbibapa]{apacite} % citas APA (da erro ao compilar en R)
\usepackage{hyperref}       % enlaces y demás
\usepackage{enumerate}      % entornos de listas
%%
%% Texto en español:
%% Comentar y descomentar segun sea.
\usepackage[spanish,activeacute, es-tabla]{babel} % textos en ES
%% 
\usepackage{multicol} % texto en varias columnas
\usepackage{fontspec} % para traballar con tipogrfías
%%
%% -----------
%% TIPOGRAFÍAS
%% -----------
%% Posibles tipografías a empregar
%% Tipografías para empregar no documento.
%% Descomentar a que se queira empregar.
%%
%\usepackage{charter}                      % Charter
%\renewcommand{\familydefault}{bch}
%\usepackage{avant}                         % Avant Grade
%\renewcommand{\familydefault}{pag}
%\usepackage{palatino}                      % Palatino
%\renewcommand{\familydefault}{ppl}
%\usepackage{helvet}                        % Helvetica
%\renewcommand{\familydefault}{phv}
%\renewcommand{\familydefault}{\sfdefault}
%\usepackage{ebgaramond}                    % EbGaramond
%\usepackage{bookman}                        % Bookman
%\usepackage{garamond}                      % Garamond
%\usepackage{nimbusserif}                    % Nimbus
%\usepackage[ttscale=0.90]{libertine} % Fuente LIBERTINE escalada Linux
%\usepackage[proportional]{libertine} % Fuente LIBERTINE Linux
%\usepackage{heuristica} % Fuente HEURISTICA
%\usepackage[heuristica,vvarbb,bigdelims]{newtxmath} %% comentar porque da ERROR
%\usepackage[T1]{fontenc}  % comentar ou descomentar según tipo de letra
%% ------------------------------------------------------------
%%
%% Fontes tipográficas empregadas neste documento:
%% -----------------------------------------------
\usepackage{libertinus}                 % Libertinus
\usepackage{libertinust1math}           % Símbolos matemáticos
%\usepackage[T1]{fontenc}
%\renewcommand*\oldstylenums[1]{\textsf{#1}}
%\usepackage{lettrine}
%\renewcommand{\ttfamily}{\fontencoding{OT1}\fontfamily{cmtt}\selectfont}

%% --------------------------
%% INTERLINEADO E INDENTADO
%% --------------------------
%% Margenes del documento
\usepackage[top=1in,bottom=1in,left=1in,right=1in]{geometry}
\setlength\parindent{0.3in} % indent at start of paragraphs (set to 0.3?)
\setlength{\parskip}{5.5pt} % separación entre párrafos
%%%\usepackage[width=150mm,top=25mm,bottom=25mm,bindingoffset=6mm]{geometry}
%%%\setlength{\parskip}{9pt}
%%
%% Interlineado
%\renewcommand{\baselinestretch}{1.25} %Interlineado 1.25 puntos (con Libertine)
%\renewcommand{\baselinestretch}{1.50} %Interlineado 1.25 puntos (con Libertine)
\renewcommand{\baselinestretch}{1.15} %Interlineado 1.15 puntos (con Heuristica)
%%
%%

%%%------------------------------------------------------------------------
%%% Formato de encabezado y pié de página
%%%------------------------------------------------------------------------
%%% Nota: este encabezamiento y el pié de página están actualizados a una nueva versión.
%%% Se comenta el anterior encabezamiento por si deseamos volver a versión anterior.

\usepackage{fancyhdr}
\pagestyle{fancy}

\renewcommand{\chaptermark}[1]{\markboth{\textit{\thechapter. #1 }}{}} % Encabezado con capítulo en cursiva (pág. izquierda)
\renewcommand{\sectionmark}[1]{\markright{\textit{\thesection\ #1}}} % Encabezado con sección en cursiva (pág. derecha)
%\renewcommand{\sectionmark}[1]{\markright{\thesection\ #1}}
\fancyhf{}
%% Encabezado
\fancyhead[RO]{\rightmark} % Encabezado a la derecha
\fancyhead[LE]{\leftmark} % Encabezado a la izquierda

%% Pié de página
\fancyfoot[RO]{\thepage} % Número de página a la derecha
\fancyfoot[LE]{\thepage} % Número de página a la izquierda

%\fancyhead[RO]{\bfseries\rightmark} % Encabezado en negrita á dereita
%\fancyhead[LE]{\bfseries\leftmark} % Encabezado en negrita á esquerda
%\fancyfoot[C]{\thepage} % Pié de página=número de página centrado

\renewcommand{\headrulewidth}{0.5pt} % Liña para encabezamentos
%\renewcommand{\footrulewidth}{0.1pt} % Línea para pés de páxina
%
\addtolength{\headheight}{0.5pt}
\fancypagestyle{plain}{
  \fancyhead{}
  \renewcommand{\headrulewidth}{0pt}
}
%
%%%------------------------------------------------------------------------
%%% Numeración de páxinas
%%%------------------------------------------------------------------------
{\newpage\renewcommand{\thepage}{\arabic{page}}\setcounter{page}{1}}

%%%------------------------------------------------------------------------
%%% ENCABEZAMENTOS E CAPÍTULOS (TEMAS - UNIDADES)
%%%------------------------------------------------------------------------
% Ver en: http://aristarco.com.es/recetario-latex/trucos-y-consejos/cambiar-titulos-partes-capitulos
% Modificamos para que non saia "Parte X" nin "Capítulo X" e que non saia "Índice Xeral"
%
\addto\captionsspanish{
\renewcommand{\partname}{BLOQUE} % Cambio Parte BLOQUE
\renewcommand{\chaptername}{TEMA} % Cambio CAPÍTULO por TEMA en mayúsculas %
\renewcommand{\contentsname}{ÍNDICE} % Cambio Índice General por "Índice" %
}
\usepackage{titlesec}
%%
%% Primeira páxina para os temas
%\newcommand{\bigrule}{\titlerule[0.25mm]} % Liña de 0.25 para capítulos
\titleformat{\chapter}[display] % Cambio formato de capítulos
{\bfseries\Huge} % por defecto se usarán caracteres de tamaño \Huge en negrita
{% contenido de la etiqueta
%\titlerule % línea horizontal superior
%\filcenter %texto centrado
%\filleft % texto alineado a la derecha (espacio en blanco izquierda)
\filright % texto alineado a la izquierda (espacio en blanco derecha)
\Huge\chaptertitlename\ % "Capítulo" o "Apéndice" en tamaño \Large en lugar de \Huge
\Huge\thechapter} % número de capítulo en tamaño \Large
{0mm} % espacio mínimo entre etiqueta y cuerpo
{\filright} % texto del cuerpo alineado a la derecha
%{\filcenter} % texto del cuerpo alineado al centro
%{\filleft} % texto del cuerpo alineado a la izquierda
%[\vspace{0.5mm} \bigrule] % después del cuerpo, dejar espacio vertical y trazar línea horizontal gruesa
%
%%%------------------------------------------------------------------------
%% FIN DE PREÁMBULO DE LATEX
%%